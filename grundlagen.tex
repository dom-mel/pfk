
\section{Fachliche Grundlagen}

Dieses Kapitel beschäftigt sich vor allem mit Quellcode.
Wie er erstellt wird, wie er aussieht und welche Entwicklungen es dabei gegeben hat.
Danach kommen wir zu ansätzen Quellcodes zu strukturieren und ihn lesbar zu gestallten.

\subsection{Quellcode und der Author}

Quelltext ist eine für den Computer verständliche Form
eines arbeitsablaufes. Er kann durch einen Kompiler in Maschienencode übersetzt werden.
In den meisten fällen ist er in Textform verfasst und für einen Menschen lesbar.

In einem Softwareprojekt ist er die detailierteste Spezifikation der zu erstellenden
Software und all ihrer Komponenten. Alles wir in ihm so eindeutig beschrieben
das ein Computer diesen umsetzen kann. Die Modell getriebene Entwicklung, vor allem
UML, ist für die direkte Übersetzung in Maschienencode aus heutige Sicht nicht geeignet
da an dieser stelle implementierungsdetails verloren gehen. \cite[S. 26]{Martin}

Der Author des Quelltextes ist der Programmierer. Seine Aufgabe ist es
die Spezifikation zu schreiben.

\subsection{Wandel in der Entwicklung von Quelltexten}

In den anfängen des Computerzeitalters war der Programmierer meistens der einzige
Leser seiner Programme. Hauptaugenmerk lag auf der Funktion es Programmes
und dem effizienten Umgang mit den Ressourcen. Damals gab es noch keine
Objekt orientierten Sprachen. Zudem waren die Arbeitsumgebungen der Entwickler
wesentlich bescheidener als in der heutigen Zeit. Quelltexte wurden im ASCII-Code
Kodiert und die Bildschirme waren nur 80 Zeichen breit.

Durch technische Entwicklungen wie z.B. das Internet, größere Bildschirme und
immer leistungsfähigere Computer. Mitlerweile gibt es für die meisten Sprachen
eine sehr gute Toolunterstützung für den Programmierer. Moderne IDEs gehen 
mitlerweile über Syntaxhighlighting hinaus. Funktionen wie automatische Code
Vervollständigung gehören mitlerweile zum Standard.

Das Internet bietet dem Entwickler eine schnelle Möglichkeit nach Lösungen und
ansätzen für Probleme oder einfach der entsprechenden Dokumentation zu suchen.

\subsection{Formatierung von Quelltexten}

Die Formatierung eines Quellcodes mit korrekter Syntax wird durch
sogenannte Whitespaces (nicht sichtbare Zeichen) vorgenommen.
Dazu gehören vorallem Tabulatoren, Leerzeichen und Zeilenumbrüche.

\begin{listing}[H]
    \begin{minted}{c}
#include <stdio.h>
int main(void){printf("Hello World!");return 0;}
    \end{minted}
    \caption{\enquote{Hello World} Programm in C mit minimalen Whitespaces}
    \label{grundlagen:hellocminimal}
\end{listing}

Für einen Compiler machen Whitespaces häufig keinen Unterschied.
Es ist egal ob der Quellcode viele Whitespaces enthält (Listing \ref{grundlagen:helloc})
oder nur minimale (\ref{grundlagen:hellocminimal}).
Der Zeilenumbruch in Zeile 1 im Listing \ref{grundlagen:helloc} ist hier Teil der Syntax.

Für den Entwickler können Whitespaces zur lesbarkeit beitragen. Listing \ref{grundlagen:helloc}
ist für einen Menschen besser verständlich, da sich mit den Whitespaces die Struktur abbilden lässt.

\begin{listing}[H]
    \begin{minted}{c}
#include <stdio.h>

int main(void) {
    printf("Hello World!");
    return 0;
}
    \end{minted}
    \caption{\enquote{Hello World} Programm in C mit Whitespaces}
    \label{grundlagen:helloc}
\end{listing}

Durch Whitespaces lässt sich der Grauwert des Quellcodes beeinflussen.
Der Grauwert ist u.a. ausschlaggebend für den Gesamteindruck und die Lesbarkeit
eines Textes \cite{Beinert}.

\subsection{Namensgebung}

Neben der Syntax gibt es viele Elemente die der Programmierer benennen muss.
Hierzu gehören z.B. Variablen, Funktionen, Methoden, Klassen, Dateien ...
Je nach Programmiersprache stehen hier unterschiedliche Zeichen zur Verfügung.
Die Buchstaben A-Z sowie die Ziffern 0-9 gehören fast immer dazu, sowie die
Sonderzeichen \enquote{-} und \enquote{\_}. Bei modernen Sprachen die UTF-8
zum Codieren verwenden ist es teilweise auch möglich Umlaute zu verwenden.

Für den Computer sind die Namen egal. So genügt es z.B. alle Variablen \enquote{a1},
\enquote{a2}, ..., \enquote{aN} zu benennen. Der Entwickler kann die Namen jedoch interprettieren und
anhand der namen ihre Bedeutung erahnen.

Das Beispieles aus \cite[S. 46-47]{Martin} zeigt, dass Variablennamen dem
Leser helfen können den Code zu interpretieren.
Listing \ref{grundlagen:namingbad} zeigt einen Quelltext deren Funktionsweise
der Leser zwar versteht aber nicht was damit bezweckt wird.

\begin{listing}
    \begin{minted}{java}
public List<int[]> getThem() {
    List<int[]> list1 = new ArrayList<int[]>();
    for (int[] x : theList)
        if (x[0] == 4)
            list1.add(x);
    return list1;
}
    \end{minted}
    \caption{1. Beispiel zu Codenamen aus \cite[S. 46]{Martin}}
    \label{grundlagen:namingbad}
\end{listing}

Listing \ref{grundlagen:naminggood} ist der selbe Quelltext zu sehen.
Es wurde die selbe Einrückung und der selbe Ablauf verwendet. Lediglich
die Variablennamen wurden verändert.

\begin{listing}
    \begin{minted}{java}
public List<int[]> getFlaggedCells() {
    List<int[]> flaggedCells = new ArrayList<int[]>();
    for (int[] cell : gameBoard)
        if (cell[STATUS_VALUE] == FLAGGED)
            flaggedCells.add(cell);
    return flaggedCells;
}
    \end{minted}
    \caption{2. Beispiel zu Codenamen aus \cite[S. 47]{Martin}}
    \label{grundlagen:naminggood}
\end{listing}

Anhand der Namen kann der Leser nun erkennen das die Methode markierte Felder eines Spielfeldes zurückgibt.

\subsection{Codeing Conventions}

Um eine einheitliche Formatierung des Quelltextes zu erreichen werden häufig
Vorgaben für die Formatierung des Codes bestimmt. Diese Vorgaben werden
\enquote{Codeing Conventions} genannt. Meist behandeln sie die Einrückung von Quelltexten,
Positionierung der Block-klammern, schreibweiße von Variablen und Methoden aber auch
welche Zeichencodierung für den Quelltext verwendet werden soll.

Beispiele für Codeing Conventions sind die GNU Coding Standards \cite{GNUCode},
der Linux Kernel Coding Style\cite{KernelCode} oder die Java Code Conventions\cite{javacode}.

Der Inhalt und Umfang von Codeing Conventions kann sich dabei stark voneinander
unterscheiden. Der Linux Kernel Coding Style ist z.B. lediglich 4 Seiten lang,
während die GNU Coding Standards 86 Seiten umfassen.

Codeing Conventions können für verschiedene Bereiche eingesetzt werden. Sie können für
ein Projekt erstellt werden wie der Linux Kernel Coding Style oder gleich für eine
ganze Programmiersprache wie die Java Code Conventions.

Eine Codeing Convention soll dafür sorgen das ein Quelltext einfach lesbar ist
und er von anderen Entwicklern schnell verstanden und verändert werden kann.

\subsection{Tools zum prüfen auf Codeing Conventions}

Wenn man einen Syntaxfehler in seinem Programm hat wird
der Kompiler beim kompilieren eine Fehlermeldung ausgeben.
Auf diese Weiße lässt sich feststellen ob der Entwickler
die Syntaktischen regeln der Programmiersprache befolgt hat.

Der Kompiler findet aber keine verstöße gegen eine bestehende Codeing Convention.
Hierfür gibt es Tools, die den Quellcode anhand von definierten Regeln analysieren
und so verstöße gegen eine Codeing Convention automatiesiert zu finden.

Ein Beispiel für ein solches Tool ist Checkstyle\footnote{siehe auch: http://checkstyle.sourceforge.net/}.
Es ist auf Java Quellcode spezialisiert und bietet eine Beispielkonfiguration für
die Java Codeing Conventions.

\subsection{Programmarten}

 - kleine "versuche" / lehre
 - programme für einmalige benutzung
 - (wachsende) anwendungen 

\subsection{Schlechter Quellcode}

 - auswirkungen
   - produktivität
 - bugs / side effects
 - komplex / schwer zu verstehen
 - lösung -> redesign?

\subsection{Guter Quellcode}

 - erfahrungssache
 - codeconventions tragen bei
 - struktur

 - Knuth
 - clean code

