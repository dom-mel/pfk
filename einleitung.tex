\section{Einleitung}
Wenn man sich Quelltexte von anderen Autoren ansieht oder eigene Quelltexte, die man vor langer Zeit geschrieben hat, wird man feststellen, dass sich der schreibstiel stark unterscheiden kann. Man merkt, dass einige Quelltexte leichter zu verstehen sind als andere auch wenn die Thematik ähnlich komplex ist. Mit dieser Lesbarkeit von Quelltexten haben sich schon viele erfahrene Programmierer beschäftigt\cite{Knuth, Heusser, Kamp, Martin, reed} aber keine von ihnen hat eine goldene Regel gefunden wie man einen leicht verständlichen Quelltext schreibt. Genauso wenig gibt es eine Metrik mit der sich die Lesbarkeit verlässlich bestimmen lässt.

In dieser Arbeit werden die Paper \enquote{Coding Guidelines: Finding the Art in the Science} von Robert Green und Henry Ledgard\cite{Green}, sowie das Paper \enquote{Reading, Writing, and Code} von Diomidis Spinellis\cite{Spinellis} betrachtet. Beide Paper behandeln das Schreiben von Quelltext. Dabei stehen keine komplexen technischen Herausforderungen im Vordergrund sondern allein der Quelltext. Was steht drin, wie kann man ihn schreiben.

Zudem wird eine Coding Convention erstellt werden, der als Leitfaden für Programmierer dienen soll, einen guten Quelltext zu schreiben. Aufgrund der gesammelten Erfahrung bei der Erstellung soll festgestellt werden, in wieweit eine Coding Convention zu einem besseren Quelltext führen kann.

Häufig wird in Arbeiten das erstellen von guten Quelltexten als eine \enquote{Art}, also Kunst, gesprochen. Auch diesem Umstand soll auf den Grund gegangen werden.
