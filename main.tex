\documentclass[a4paper]{article}
\usepackage[utf8]{inputenc}
\usepackage{csquotes}
\usepackage[ngerman]{babel}
\usepackage{biblatex}
\usepackage{float}
\usepackage{graphicx}
\usepackage{epstopdf}
\usepackage{subfigure}
\usepackage{array}
\usepackage{booktabs}
\makeatletter
\renewcommand\@ptsize{13}
\makeatother
\usepackage{extsizes}
%\setcounter{secnumdepth}{-1} 
\usepackage{hyperref}
\usepackage{nameref}
\usepackage{minted}
\makeatletter
\renewcommand\paragraph{%
   \@startsection{paragraph}{4}{0mm}%
      {-\baselineskip}%
      {.5\baselineskip}%
      {\normalfont\normalsize\bfseries}}
\makeatother
\usemintedstyle{friendly}
\bibliography{bibliography}
\title{Semesterarbeit im Fach Programmieren fortgeschrittene Konzepte}
\author{Dominik Eckelmann, Matr.-Nr.: 785856}
\date{\today}

\begin{document}

\sloppy

\begin{figure}[H]
	\centering
	\includegraphics[width=0.7\textwidth]{beuth.eps}
	\maketitle
\end{figure}

\newpage
\tableofcontents

\newpage
\section{Einleitung}

In dieser Arbeit werden die Paper \enquote{Coding Guidelines: Finding the Art in the Science} von Robert Green und Henry Ledgard\cite{Green}, sowie das Paper \enquote{Reading, Writing, and Code} von Diomidis Spinellis betrachtet. 

Das Kernthema ist die art und weiße wie Quelltexte geschrieben werden. Jeder Programmierer kennt das Problem sich in einem fremden Quelltext zurechtzufinden.
\cite{Knuth}
 - code
 - syntax
 - lagecy
 - team
 - international
 - spec
 - ide's
 - naming


\section{Fachliche Grundlagen}

Dieses Kapitel beschäftigt sich vor allem mit Quellcode.
Wie er erstellt wird, wie er aussieht und welche Entwicklungen es dabei gegeben hat.
Danach kommen wir zu ansätzen Quellcodes zu strukturieren und ihn lesbar zu gestallten.

\subsection{Quellcode und der Author}

Quelltext ist eine für den Computer verständliche Form
eines arbeitsablaufes. Er kann durch einen Kompiler in Maschienencode übersetzt werden.
In den meisten fällen ist er in Textform verfasst und für einen Menschen lesbar.

In einem Softwareprojekt ist er die detailierteste Spezifikation der zu erstellenden
Software und all ihrer Komponenten. Alles wir in ihm so eindeutig beschrieben
das ein Computer diesen umsetzen kann. Die Modell getriebene Entwicklung, vor allem
UML, ist für die direkte Übersetzung in Maschienencode aus heutige Sicht nicht geeignet
da an dieser stelle implementierungsdetails verloren gehen. \cite[S. 26]{Martin}

Der Author des Quelltextes ist der Programmierer. Seine Aufgabe ist es
die Spezifikation zu schreiben.

\subsection{Wandel in der Entwicklung von Quelltexten}

In den anfängen des Computerzeitalters war der Programmierer meistens der einzige
Leser seiner Programme. Hauptaugenmerk lag auf der Funktion es Programmes
und dem effizienten Umgang mit den Ressourcen. Damals gab es noch keine
Objekt orientierten Sprachen. Zudem waren die Arbeitsumgebungen der Entwickler
wesentlich bescheidener als in der heutigen Zeit. Quelltexte wurden im ASCII-Code
Kodiert und die Bildschirme waren nur 80 Zeichen breit.

Durch technische Entwicklungen wie z.B. das Internet, größere Bildschirme und
immer leistungsfähigere Computer. Mitlerweile gibt es für die meisten Sprachen
eine sehr gute Toolunterstützung für den Programmierer. Moderne IDEs gehen 
mitlerweile über Syntaxhighlighting hinaus. Funktionen wie automatische Code
Vervollständigung gehören mitlerweile zum Standard.

Das Internet bietet dem Entwickler eine schnelle Möglichkeit nach Lösungen und
ansätzen für Probleme oder einfach der entsprechenden Dokumentation zu suchen.

\subsection{Formatierung von Quelltexten}

Die Formatierung eines Quellcodes mit korrekter Syntax wird durch
sogenannte Whitespaces (nicht sichtbare Zeichen) vorgenommen.
Dazu gehören vorallem Tabulatoren, Leerzeichen und Zeilenumbrüche.

\begin{listing}[H]
    \begin{minted}{c}
#include <stdio.h>
int main(void){printf("Hello World!");return 0;}
    \end{minted}
    \caption{\enquote{Hello World} Programm in C mit minimalen Whitespaces}
    \label{grundlagen:hellocminimal}
\end{listing}

Für einen Compiler machen Whitespaces häufig keinen Unterschied.
Es ist egal ob der Quellcode viele Whitespaces enthält (Listing \ref{grundlagen:helloc})
oder nur minimale (\ref{grundlagen:hellocminimal}).
Der Zeilenumbruch in Zeile 1 im Listing \ref{grundlagen:helloc} ist hier Teil der Syntax.

Für den Entwickler können Whitespaces zur lesbarkeit beitragen. Listing \ref{grundlagen:helloc}
ist für einen Menschen besser verständlich, da sich mit den Whitespaces die Struktur abbilden lässt.

\begin{listing}[H]
    \begin{minted}{c}
#include <stdio.h>

int main(void) {
    printf("Hello World!");
    return 0;
}
    \end{minted}
    \caption{\enquote{Hello World} Programm in C mit Whitespaces}
    \label{grundlagen:helloc}
\end{listing}

Durch Whitespaces lässt sich der Grauwert des Quellcodes beeinflussen.
Der Grauwert ist u.a. ausschlaggebend für den Gesamteindruck und die Lesbarkeit
eines Textes \cite{Beinert}.

\subsection{Namensgebung}

Neben der Syntax gibt es viele Elemente die der Programmierer benennen muss.
Hierzu gehören z.B. Variablen, Funktionen, Methoden, Klassen, Dateien ...
Je nach Programmiersprache stehen hier unterschiedliche Zeichen zur Verfügung.
Die Buchstaben A-Z sowie die Ziffern 0-9 gehören fast immer dazu, sowie die
Sonderzeichen \enquote{-} und \enquote{\_}. Bei modernen Sprachen die UTF-8
zum Codieren verwenden ist es teilweise auch möglich Umlaute zu verwenden.

Für den Computer sind die Namen egal. So genügt es z.B. alle Variablen \enquote{a1},
\enquote{a2}, ..., \enquote{aN} zu benennen. Der Entwickler kann die Namen jedoch interprettieren und
anhand der namen ihre Bedeutung erahnen.

Das Beispieles aus \cite[S. 46-47]{Martin} zeigt, dass Variablennamen dem
Leser helfen können den Code zu interpretieren.
Listing \ref{grundlagen:namingbad} zeigt einen Quelltext deren Funktionsweise
der Leser zwar versteht aber nicht was damit bezweckt wird.

\begin{listing}
    \begin{minted}{java}
public List<int[]> getThem() {
    List<int[]> list1 = new ArrayList<int[]>();
    for (int[] x : theList)
        if (x[0] == 4)
            list1.add(x);
    return list1;
}
    \end{minted}
    \caption{1. Beispiel zu Codenamen aus \cite[S. 46]{Martin}}
    \label{grundlagen:namingbad}
\end{listing}

Listing \ref{grundlagen:naminggood} ist der selbe Quelltext zu sehen.
Es wurde die selbe Einrückung und der selbe Ablauf verwendet. Lediglich
die Variablennamen wurden verändert.

\begin{listing}
    \begin{minted}{java}
public List<int[]> getFlaggedCells() {
    List<int[]> flaggedCells = new ArrayList<int[]>();
    for (int[] cell : gameBoard)
        if (cell[STATUS_VALUE] == FLAGGED)
            flaggedCells.add(cell);
    return flaggedCells;
}
    \end{minted}
    \caption{2. Beispiel zu Codenamen aus \cite[S. 47]{Martin}}
    \label{grundlagen:naminggood}
\end{listing}

Anhand der Namen kann der Leser nun erkennen das die Methode markierte Felder eines Spielfeldes zurückgibt.

\subsection{Codeing Conventions}

Um eine einheitliche Formatierung des Quelltextes zu erreichen werden häufig
Vorgaben für die Formatierung des Codes bestimmt. Diese Vorgaben werden
\enquote{Codeing Conventions} genannt. Meist behandeln sie die Einrückung von Quelltexten,
Positionierung der Block-klammern, schreibweiße von Variablen und Methoden aber auch
welche Zeichencodierung für den Quelltext verwendet werden soll.

Beispiele für Codeing Conventions sind die GNU Coding Standards \cite{GNUCode},
der Linux Kernel Coding Style\cite{KernelCode} oder die Java Code Conventions\cite{javacode}.

Der Inhalt und Umfang von Codeing Conventions kann sich dabei stark voneinander
unterscheiden. Der Linux Kernel Coding Style ist z.B. lediglich 4 Seiten lang,
während die GNU Coding Standards 86 Seiten umfassen.

Codeing Conventions können für verschiedene Bereiche eingesetzt werden. Sie können für
ein Projekt erstellt werden wie der Linux Kernel Coding Style oder gleich für eine
ganze Programmiersprache wie die Java Code Conventions.

Eine Codeing Convention soll dafür sorgen das ein Quelltext einfach lesbar ist
und er von anderen Entwicklern schnell verstanden und verändert werden kann.

\subsection{Tools zum prüfen auf Codeing Conventions}

Wenn man einen Syntaxfehler in seinem Programm hat wird
der Kompiler beim kompilieren eine Fehlermeldung ausgeben.
Auf diese Weiße lässt sich feststellen ob der Entwickler
die Syntaktischen regeln der Programmiersprache befolgt hat.

Der Kompiler findet aber keine verstöße gegen eine bestehende Codeing Convention.
Hierfür gibt es Tools, die den Quellcode anhand von definierten Regeln analysieren
und so verstöße gegen eine Codeing Convention automatiesiert zu finden.

Ein Beispiel für ein solches Tool ist Checkstyle\footnote{siehe auch: http://checkstyle.sourceforge.net/}.
Es ist auf Java Quellcode spezialisiert und bietet eine Beispielkonfiguration für
die Java Codeing Conventions.

\subsection{Programmarten}

 - kleine "versuche" / lehre
 - programme für einmalige benutzung
 - (wachsende) anwendungen 

\subsection{Schlechter Quellcode}

 - auswirkungen
   - produktivität
 - bugs / side effects
 - komplex / schwer zu verstehen
 - lösung -> redesign?

\subsection{Guter Quellcode}

 - erfahrungssache
 - codeconventions tragen bei
 - struktur

 - Knuth
 - clean code




\section{Coding Guidelines: Finding the Art in Science}
Das erste Paper, dass in dieser Arbeit behandelt wird, trägt den Titel \enquote{ Coding Guidelines: Finding the Art in Science, What separates good code from great code?} von Robert Green und Henry Ledgard\cite{Green}. Die Autoren versuchen eigene Regeln, für das entwickeln von Quelltext aufzustellen. Das Ziel ist es den Programmierer in die Lage zu versetzen, einen Quelltext zu schreiben der schnell und einfach von anderen Programmierern gelesen, verstanden und erweitert werden kann. Das schließt mit ein, dass sich der Entwickler, ohne Kenntnis des Systems, schnell im gesamten Quelltext zurechtfinden kann. Dies ist vor allem in großen, langlebigen Softwareprojekten von Nöten, da hier eine gewisse Fluktuation bei den Entwicklern nicht auszuschließen ist\ref[S. 12]{Green}.

Beim erstellen der Regeln wird nicht auf bestimmte Entwicklungsumgebungen oder andere Tools Rücksicht genommen. Die Regeln sollen mit allen Editoren angewandt werden können.
\subsection{Monospace and the ASCII}
Als erstes wird bezug auf den Artikel von Kamp\cite{Kamp} genommen. Dieser stellte fest wie wenig sich das Aussehen von Quelltexten im Laufe der Zeit geändert hat und wie sehr es noch dem Schreiben mit einer Schreibmaschine ähnelt. Es wurden zwar viele nützliche Programme entwickelt die beim erstellen von Quelltexten helfen können, trotzdem ist er immer noch in Textform. Noch dazu wird immer eine Monospace Schriftart zur Darstellung verwendet. Dazu kommt noch das es kein klassischer Fließtext ist. Der verwendete Wortschatz ist stark eingeschränkt und es werden häufig mathematische Darstellungsformen gewählt \cite[S. 2]{Green}. Darum wir im Paper\cite{Green} vorgeschlagen für Quelltext eine tabellenartige Formatierung zu wählen um die Lesbarkeit zu erhöhen. Ein Beispiel hierfür ist in Listing \ref{paper1:table} zu sehen. Die \enquote{\texttt{case}}, sowie die \enquote{\texttt{break}} Anweisungen befinden sich in einer Spalte und die Aufrufe für die einzelnen Verzweigungspfade befinden sich in einer Spalte.

\begin{listing}[H]
    \begin{minted}{c}
char c1;
c1 = getChoice();
switch(c1) {
	case 'q': case 'Q': quit();                 break;
	case 'e': case 'e': enterPerson(content);   break;
	case 'd': case 'd': delPerson(content);     break;
	case 's': case 's': sortByName();           break;
	case 'l': case 'l': showAll();              break;
	case 'f': case 'f': searchByName(content);  break;
	case default: System.out.println(
		"--Invalid Command!!\n");
}
    \end{minted}
    \caption{Beispiel für tabellarische Darstellung von Quelltext aus \cite[S. 2]{Green}}
    \label{paper1:table}
\end{listing}



\ref{grundlagen:namensgebung}

\subsection{Naming}
Als nächstes wenden Sich Robert Green und Henry Ledgard der Benennung von Elementen zu\cite[S. 3f.]{Green}. Die Benennung aller Elemente ist ein wichtiges belang bei der Entwicklung. Eine Sinnvolle Benennung hilft dem Leser eines Quelltextes sich in diesem zu Orientieren. Sie stellen hierzu die folgenden Regeln auf\cite[S. 4]{Green}:
\begin{itemize}
	\item Variablen- und Klassennamen sollen Nomen bzw. Nominalphrasen sein
	\item Klassennamen sind zusammenfassende Nomen (\texttt{Book})
	\item Variablennamen sind exakte beschreibende Nomen (\texttt{BookingNumber})
	\item Prozeduren sollten Verben bzw. Verbphrasen sein (\texttt{CalculateCost})
	\item Wertliefernde Methoden sollen Nomen Phrasen sein (\texttt{GetName}) 
	\item Boolesche Werte sollen Adjektive sein
	\item Für zusammengesetzte Namen sollte man sich an die englische Sprache halten
	\item Namen sollten aussprechbar sein
\end{itemize}

Auf diese weiße soll dem Leser zu jeder zeit klar sein, was hinter einem Namen steht. 


\section{Reading, Writing, and Code}
Das zweite Paper, welches in dieser Arbeit behandelt wird ist \enquote{Reading, Writing, and Code} von Diomidis Spinellis\cite{Spinellis}. In diesem Abschnitt werden die wesentlichen Teile der Arbeit zusammengefasst. Das Paper beschreibt die Schwierigkeiten, Quelltext zu schreiben, der von einfach zu lesen und verstehen ist. 

Spinellis geht zunächst von der Annahme aus, dass Quelltext wesentlich einfacher zu schreiben, als zu verstehen ist. Dies begründet er damit, dass es zur Lösung eins Problems viele verschiedene Lösungswege gibt. Der Autor ist sich beim Schreiben immer bewusst, welchen Weg er geht, während der Leser sich den Weg erst erschließen muss \cite[S. 85]{Spinellis}. Guter Quelltext zeichnet sich an dieser Stelle dadurch aus, dass der Leser den Gedanken des Autors gut nachvollziehen kann. Das Schreiben von gutem Quelltext nimmt aber meist mehr Zeit in Anspruch. Ferner zieht er den Schluss, dass sich guter Quelltext erst in der Zukunft, sprich bei der Wartung und Erweiterung des Quelltextes auszahlt\cite[S. 86]{Spinellis}. Hier bemängelt er auch, dass in der Lehre zu wenig auf das Schreiben von Programmen in einem realen Umfeld eingegangen wird. Es ist meist so, dass der Quelltext von einer Person, in einer Programmiersprache, für genau eine Zielplattform entwickelt wird. In einer realen Umgebung entwickelt aber mehrere Entwickler an einem Quelltext, der in verschiedenen Programmiersprachen entwickelt und auf verschiedenen Plattformen lauffähig sein muss. \cite[S. 86]{Spinellis}

Nach Spinellis gehören zu den am schwersten zu verstehenden Quelltexten zum einen Systemnahe Programme, wie Datenbank Systeme, Grafikengines, Betriebssystem Kernel, etc., und zum anderen Objekt-Orientierte-Programme, die eine unangemessene abstraktionstiefe besitzen. \cite[S. 86]{Spinellis}

Seine Erkenntnis ist es, dass es keinen einfachen Weg gibt, gut verständlichen Quelltext zu schreiben. Um einen solchen Quelltext zu schreiben benötig der Autor vor allem Erfahrung. Er muss wissen wie man einen Quelltext schreibt, auch mithilfe von Coding Conventions, und wann man die Coding Conventions brechen soll. \cite[S. 86]{Spinellis}

\subsection{Programmiersprachen und sauberer Quelltext}
Für Spinellis kann die Grammatik einer Programmiersprache dazu beitragen gut lesbaren Quelltext zu schreiben. Als Beispiele für diese Kategorie führt er C++, Java, Ada, und Perl an. Mit Fortran 77 lässt sich genauso guter Quelltext schreiben, es ist aber im Gegensatz zu den vorher aufgeführten Programmiersprachen aufwändiger. \cite[S. 87]{Spinellis}

Dazu kommt noch das einige Sprachen, wie C++, dem Entwickler Sprachfeatures zur Verfügung stellen die den Quelltext schnell unleserlich werden lassen können, \enquote{[...]there are languages that discourage you from writing bad code through the lack of \enquote{dangerous} features, and there are languages that give you more than enough rope to hang yourself and all your code’s future maintainers.}\cite[S. 87]{Spinellis}

Zu diesen Features zählt er u.a.:
\begin{enumerate}
\item die GoTo-Anweisung
\item Operatorüberladung
\item Pointer
\item Dynamische Speicherverwaltung 
\item C/C++ Präprozessor Makros
\end{enumerate}

Diese können den Leser mit Leichtigkeit in die Irre führen.

\subsection{Kommentare}
Kommentare sind für Spinellis ein wichtiges Werkzeug, um die Lesbarkeit eines Quelltextes zu erhöhen. Damit ein Kommentar jedoch wertvoll für den Leser wird, muss dieser klar formuliert sein. Sie können dem Leser so eine Hilfe beim verstehen des Zusammenhanges sein. Einfach die Bedeutung der nächsten Anweisung wiedergeben steigert nicht die Lesbarkeit. Zum anderen lassen sich auf diese Weiße automatisch technische Dokumentationen generieren, als Beispiel nennte er JavaDoc\footnote{s.a. http://www.oracle.com/technetwork/java/javase/documentation/index-jsp-135444.html}. Damit die Kommentare jedoch gut genug sind muss das erstellen dieser direkt beim Schreiben des Quelltextes geschehen. Zu einem späteren Zeitpunkt erstellte Kommentare sind häufig qualitativ weniger Wert. \cite[S. 88]{Spinellis}

\subsection{Vorschläge zum Schreiben von besserem Quelltext}
Um nun als Autor einen bessere Quelltexte zu schreiben, gibt Spinellis den Rat keine IDE zu verwenden, sondern einen Texteditor, wie VIM oder EMACS. Er begründet dies damit, dass ein Entwickler, der eine IDE verwendet, nur aus der Sichtweiße seiner IDE schreibt. Einem Leser mit einer anderen IDE oder einem Texteditor, kann es so erschwert werden den Quelltext zu lesen. Zudem sollte beim Einsatz einer IDE auch sichergestellt werden, dass der Quelltext nicht in einem Binärformat abgelegt wird. Nur dann kann dieser in einem Texteditor sinnvoll angezeigt werden. Als Nebeneffekt lässt sich ein Quelltext in Textform besser in einem Versionsverwaltungssystem ablegen. \cite[S. 88]{Spinellis}




\section{Eigener Ansatz}

In diesem Teil der Arbeit wird eine Coding Convention erstellt. Sie soll dafür sorgen, dass ein Quelltext, der von mehreren Entwicklern geschrieben wird, eine Einheitliche Form annimmt. Zudem soll durch den Einsatz der Coding Convention die Lesbarkeit und Verständlichkeit des Quelltextes gefördert werden. Als Zielprogrammiersprache wird Java verwendet. In den hier erstellten Code Conventions wird kein Wert darauf gelegt, die einzig richtige Vorlage für guten Quelltext zu sein. 


\subsection{Formatierung}
Zunächst einige allgemeine Festlegungen zur Formatierung des Quelltextes. Diese dienen dazu, ein einheitliches Textbild zu erhalten.

Der Zeichensatz der bei der Entwicklung verwendet werden muss ist UTF-8. Dieses wird von der Programmiersprache selbst vorgegeben. Durch die Verwendung, des UTF-8 Zeichensatzes, können die meisten länderspezifischen Umlaute und Sonderzeichen abgebildet werden. Des weiteren können so Umlautfehler vermieden werden.

Für die Einrückung werden vier Leerzeichen verwendet. Der Einsatz von Tabulatoren hätte den Vorteil, dass jeder Entwickler die Tiefe der Einrückung nach seinen Wünschen einstellen kann. Auf den Einsatz von Tabulatoren ist aber verzichtet worden, weil diese zu überlangen Zeilen führen können. Wenn ein Entwickler einen Quelltext schreibt und bei ihm ein Tabulator die Länge von zwei Leerzeichen hat, ist es für ihn möglich sehr viel längere Zeilen zu schreiben als ein Entwickler dessen Tabulatoren acht Leerzeichen lang sind ohne gegen die nächste Regel im bezug auf Zeilenlänge zu verstoßen.

Eine Zeile sollte nicht breiter als 120 Zeichen werden. Sollte eine Zeile länger werden kann diese an geeigneter Stelle umgebrochen werden. Die Fortsetzung wird doppelt, sprich acht Leerzeichen tief, eingerückt.

\subsubsection{Blöcke}
Der Inhalt von Blockklammern wird immer eingerückt. Die öffnende Blockklammer wird immer in derselben Zeile verwendet wie die dazugehörige Anweisung. Die schließende steht für sich alleine. Optionale Blockklammern bei Bedingungen und Schleifen sind immer zu setzen. Wenn es Sinnvoll ist, können Blockklammern zur Verdeutlichung des Quelltextes eingesetzt werden. In Listing \ref{own:blockopengl}

\begin{listing}[H]
    \begin{minted}{java}
gl.PushMatrix();
{
    gl.glTranslatef(width/2, height/2, 0); 
    gl.glRotatef(a, 0, 0, 0); 

    gl.glBegin(GL.GL_TRIANGLES); 
    gl.glColor4f(0.7, 0.1, 0.7, 0.8); 
    gl.glVertex3f(0, 0, 0); 
    gl.glVertex3f(0, 50, 0); 
    gl.glVertex3f(25, 0, 25); 
    gl.glEnd(); 
}
gl.PopMatrix();
    \end{minted}
    \caption{Einsatz von Blockklammern zur Verdeutlichung der Struktur am Beispiel von OpenGL}
    \label{own:blockopengl}
\end{listing}


Die Transformationsoperationen nach dem \enquote{\texttt{gl.PushMatrix}} Aufruf gelten lediglich bis zum entsprechenden nächsten \enquote{\texttt{gl.PopMatrix}} Aufruf. Da die angewandten Rotationen und Translationen lediglich zwischen den aufrufen von Bedeutung sind kann es zur Lesbarkeit beitragen diese in einen extra Block zu schreiben.

\subsubsection{Zeilenumbrüche und Leerzeilen}
Als Zeilenumbruch wird der UNIX Zeilenumbruch verwendet. Dies bedeutet lediglich ein Zeilenvorschub(\textbackslash n), ohne Wagenrücklauf(\textbackslash r).

Dieser ist nach jeder öffnenden Blockklammer \enquote{\{} und jedem Semikolon zwingend. Durch diese Vorgabe wird sichergestellt, dass jede Anweisung in einer eigenen Zeile steht. Dies gilt genauso für Annotationen. Nach jeder Annotation muss ein Zeilenumbruch eingefügt werden. In \texttt{\enquote{ENUM}s} muss nach jedem Kommata ein Zeilenumbruch sein.

Leerzeilen sollen dazu verwendet werden eine sichtbare Trennung zwischen logischen Blöcken zu erzeugen. Die Trennung sollte jedoch maximal aus einer Leerzeile bestehen. Daraus ergeben sich u.a. Leerzeilen an folgenden Positionen:

\begin{itemize}
\item Nach der \texttt{package} Deklaration
\item Ggf. zwischen Importanweisungen von verschiedenen Quellen
\item Nach den Importen
\item Ggf. zwischen der Deklaration von Klassenvariablen
\item Zwischen der Deklaration von Klassenvariablen und Methoden
\item Zwischen den Methoden, inneren Klassen und inneren \enquote{\texttt{ENUM}s}
\end{itemize}

Diese sollen die Lesbarkeit des Quelltextes erhöhen. Der Quelltext wirkt dadurch weniger komprimiert. Zudem sollen die Leerzeilen logische zusammenhänge hervorheben.

\subsection{Leerzeichen}

Um eine bessere Lesbarkeit einzelner Zeilen zu erlangen werden an bestimmten Stellen Leerzeichen eingefügt. Dieses geschieht analog zu den Vorschlägen von Green und Ledgard\cite[S. 7]{Green}. Demnach wird nach jedem Komma ein Leerzeichen eingefügt. Zudem vor und nach nicht unären Operatoren. Der Memberoperator \enquote{.} bildet ist hiervon ausgenommen. Beispiele sind in Listing \ref{paper1:spaces}, auf Seite \pageref{paper1:spaces} zu sehen.




\section{Fazit}

Einen guten Quelltext zu schreiben ist eine Komplizierte Angelegenheit. Es reicht nicht aus, dass der Quelltext lediglich die gestellte Aufgabe löst. Er muss vielmehr durch einen klaren Schreibstil und eine klare Ausdrucksweise, dem Leser seine Intention in jedem Detail deutlich machen. Das gilt für sowohl für die kleinsten Implementierungsdetails, wie auch für das große Gesamtbild der Software.

Zu einem guten Quelltext gehört genauso eine angemessene Architektur, welche die abstrakte Arbeitsweiße der Software deutlich macht. Der große Unterschied zwischen Mensch und Maschine ist nun einmal, dass die Maschine lediglich den Quelltext abarbeiten muss, der Mensch ihn aber verstehen möchte.

Deshalb ist es so schwer eine gute Coding Convention zu erstellen. Es gibt keine universelle ausdrucksweiße, die jeder Mensch gleichermaßen schnell verstehen kann. Dazu kommt noch, dass beim Schreiben eines Quelltextes, der Autor, das Problem schon verstanden und einen Lösungsweg im Kopf hat. Wenn der Autor einen neuen Quelltext schreibt, muss er sich nicht mit dem Quelltext und damit den Lösungswegen von anderen Autoren herumschlagen. Er kann sich vielmehr ausschließlich auf die Problemlösung konzentrieren. In der Realität ist dies aber sehr selten der Fall. Meistens ist es so, dass an ein bereits vorhandener Quelltext weiterentwickelt bzw. gewartet wird. Dabei muss sich der Autor häufig mit dem Quelltext von anderen Autoren auseinander setzen. 

Damit ein Autor einen guten Quelltext schreiben kann, muss er selbst schon viele Quelltexte gelesen und geschrieben haben, damit er weiß, wie schwer es sein kann einen Quelltext zu verstehen.

Eine Coding Convention kann nicht alleine dafür sorgen, dass man einen guten Quelltext erhält. Sie kann lediglich als Richtlinie dienen um eine einheitliche Formatierung zu gewährleisten. Einen wirklich guten Quelltext zu schreiben gehört zu einer hohen Kunst in der Zunft der Softwareentwickler.




\appendix

\newpage

\nocite{*}
 
\printbibliography
\addcontentsline{toc}{section}{Literatur}

\newpage
 
\addcontentsline{toc}{section}{Abbildungsverzeichnis}
\listoffigures

\newpage
 
\addcontentsline{toc}{section}{Quellcodeverzeichnis}
\renewcommand\listoflistingscaption{Quellcodeverzeichnis}
\listoflistings

\end{document}
