\section{Fazit}

Einen guten Quelltext zu schreiben ist eine Komplizierte Angelegenheit. Es reicht nicht aus, dass der Quelltext lediglich die gestellte Aufgabe löst. Er muss vielmehr durch einen klaren Schreibstil und eine klare Ausdrucksweise, dem Leser seine Intention in jedem Detail deutlich machen. Das gilt für sowohl für die kleinsten Implementierungsdetails, wie auch für das große Gesamtbild der Software.

Zu einem guten Quelltext gehört genauso eine angemessene Architektur, welche die abstrakte Arbeitsweiße der Software deutlich macht. Der große Unterschied zwischen Mensch und Maschine ist nun einmal, dass die Maschine lediglich den Quelltext abarbeiten muss, der Mensch ihn aber verstehen möchte.

Deshalb ist es so schwer eine gute Coding Convention zu erstellen. Es gibt keine universelle ausdrucksweiße, die jeder Mensch gleichermaßen schnell verstehen kann. Dazu kommt noch, dass beim Schreiben eines Quelltextes, der Autor, das Problem schon verstanden und einen Lösungsweg im Kopf hat. Wenn der Autor einen neuen Quelltext schreibt, muss er sich nicht mit dem Quelltext und damit den Lösungswegen von anderen Autoren herumschlagen. Er kann sich vielmehr ausschließlich auf die Problemlösung konzentrieren. In der Realität ist dies aber sehr selten der Fall. Meistens ist es so, dass an ein bereits vorhandener Quelltext weiterentwickelt bzw. gewartet wird. Dabei muss sich der Autor häufig mit dem Quelltext von anderen Autoren auseinander setzen. 

Damit ein Autor einen guten Quelltext schreiben kann, muss er selbst schon viele Quelltexte gelesen und geschrieben haben, damit er weiß, wie schwer es sein kann einen Quelltext zu verstehen.

Eine Coding Convention kann nicht alleine dafür sorgen, dass man einen guten Quelltext erhält. Sie kann lediglich als Richtlinie dienen um eine einheitliche Formatierung zu gewährleisten. Einen wirklich guten Quelltext zu schreiben gehört zu einer hohen Kunst in der Zunft der Softwareentwickler.


